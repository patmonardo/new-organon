\section*{Fichte: Prolegomena to the Science of Knowing (1804)}

\subsection*{First Lecture (Monday, April 16, 1804)}
Honorable Guests:

Nothing about the venture which we now jointly begin
is as difficult as the beginning;
indeed, even the escape route which I just started to take
by beginning with a consideration of the difficulty of beginning
has its own difficulties.
No means remain except to cut the knot boldly
and to ask you to accept that what I am about to say
is aimed at the wide world generally
and applies to it but not at all to you.

Namely: in my view the chief characteristic of our time is
that in it life has become merely historical and symbolic,
while real living is scarcely ever found.
One not insignificant aspect of life is thinking.
Where all life has degenerated into a strange tale,
the same must also happen to thinking.
Of course one will have heard and made a note of the fact that,
among other things, human beings can think;
indeed that many of them have thought, the first in this way,
the second differently, and the third and fourth each in yet another way
and that all have failed in some fashion.
It is not easy to decide to undertake this
thought process again for oneself.
One who assumes responsibility for arousing an era like ours
to this decision must face this discomfort among others:
he doesn't know where he might look for, and find,
the people who need arousing.
Whomever he accuses has a ready answer,
“Yes, that is certainly true of others, but not of us”;
and they are right to the extent that, as well as knowing
the criticized form of thought they are also familiar historically
with the opposite schools, so that if someone attacks them on one side,
they are ready to flee to the very position they currently reject.
So, for example, if one speaks in the way I did just now,
rebuking our historical superficiality,
dispersion into the most multifarious and contradictory opinions,
indecisiveness about everything altogether, and absolute indifference
to truth in the way I have just rebuked these things,
then everyone will be sure that he does not recognize himself in this picture,
that he knows very well that only one thing can be true
and that all contradictions must necessarily be false.
If we accused him of dogmatic rigidity and one-sidedness
for his adherence to the one, the very same person
would praise his all-around skepticism.

In such a state of affairs there is nothing for it
but to assert briefly and sweetly at a single stroke
that I presuppose here in all seriousness that
there is a truth which alone is true and
everything apart from this is unconditionally false;
further, that this truth can actually be found
and be immediately evident as unconditionally true;
but that not even the least spark of it can be
grasped or communicated historically as
an appropriation from someone else's mind.
Rather, whomever would have it must
produce it entirely out of himself.
The presenter can only provide the terms for insight;
each individual must fulfill these terms in himself,
applying his living spirit to it with all his might,
and then the insight will happen of itself without any further ado.
There is no question here of an object
which is already well known from other contexts,
but of something completely new, unheard of, and totally unknown to anyone
who has not already studied the science of knowing thoroughly.
No one can arrive at this unknown unless it produces itself in him,
but it does this only under the condition that this very person
produces something, namely the conditions for insight's self-production.
Whoever does not do this will never obtain
the object about which we will speak here.
And since our whole discourse will be about this object,
he will have no object at all;
for him our entire discourse will be words about pure,
bare nothingness, an empty vessel, word breath,
the mere movement of air, and nothing more.
And so let this, taken rigorously and literally,
serve as the first prolegomenon.

I have still more to add, which however first presupposes the following.
From now on, honorable guests, I wish to be considered silenced and erased,
and you yourselves must come forward and stand in my place.
From now on, everything which is to be thought in this assembly
should be thought and be true only to the extent that
you yourselves have thought it and seen it to be true.

I said that I have more to add by way of introduction,
and I will devote this week's four lectures to this task.
Repeated experience compels me to remind you explicitly that
these introductory remarks should not be viewed as most are,
that is, as a simple approach which the lecturer takes and
whose content has no very great significance.
The introductory remarks I will present have meaning,
and what follows will be entirely lost without them.
They are designed to call your spiritual eye
away from the objects over which it has heretofore glided to and fro,
directing it to the point which we must consider here, and
indeed to give this point its existence for the very first time.
They are designed to initiate you into the art
which we will subsequently practice together,
the art of philosophy.
They should simultaneously acquaint you, and make you fluent,
with a system of rules and maxims of thinking whose
employment will recur in every lecture.

I hope the matters which are to be handled in the introduction will
become easily comprehensible to everyone of even moderate attention;
but past experience requires that I add a word about just this comprehending.
First, one must not assume that the standard of comprehensibility for
the science of knowing in general is comparable to the standard of
study and attention which the introduction requires;
since anyone assuming this will be unpleasantly disappointed later on.
Thus, whomever has heard and understood the introductory remarks
has acquired a true and fitting concept of the science of knowing
sanctioned by the very originator of this science,
but the listener has not thereby acquired
the tiniest spark of the science itself.
This universally applicable distinction between
the mere concept and the real, true substance has
particular bearing here.
Possessing the concept has its uses;
for example, it protects us from the absurdity of
underestimating and misjudging what we do not possess;
only no one should believe that by this possession,
which is not in any case all that rare,
one has become a philosopher.
One is and remains a sophist,
only to be sure less superficial than
those who do not even possess the concept.

Following these prefatory remarks about the prefatory remarks,
let's get to work.
I have promised a discourse on the science of knowing,
and science of knowing you expect.
What is the science of knowing?

First, in order to start with what everyone would admit,
and to speak of it as others would:
without doubt it is one of
the possible philosophical systems,
one of the philosophies.
So much as an initially stated genus,
according to the rules of definition.

So what generally is philosophy and what is it commonly taken to be;
or, as one could more easily say, what should it be
according to what is generally required of it?

Without doubt: philosophy should present the truth.
But what is truth, and what do we actually search for when we search for it?
Let's just consider what we will not allow to count as truth:
namely when things can be this way or equally well the other;
for example the multiplicity and variability of opinion.
Thus, truth is absolute oneness and invariability of opinion.
So that I can let go of the supplemental term “opinion,”
since it will take us too far afield, let me say that
the essence of philosophy would consist in this:
to trace all multiplicity
(which presses itself upon us in the usual view of life)
back to absolute oneness.
I have stated this point briefly;
and now the main thing is not to take it superficially,
but energetically and as something which ought
in all seriousness to hold good.
All multiplicity whatever can even be distinguished, or
has its antithesis, or counterpart, absolutely without exception.
Wherever even the possibility of a distinction remains,
whether explicitly or tacitly, the task is not completed.
Whoever can point out the smallest distinction in or
with regard to what some philosophical system has posited
as its highest principle has refuted that system.

As is obvious from this, absolute oneness is
what is true and in itself unchangeable,
its opposite purely contained within itself.
“To trace back”, precisely in the continuing insight of
the philosopher himself as follows:
that he reciprocally conceives multiplicity through oneness
and oneness through multiplicity.
That is, that, as a principle, Oneness = A illuminates
such multiplicity for him;
and conversely, that multiplicity in its ontological ground can
be grasped only as proceeding from A.

The science of knowing has this task in common with all philosophy.
All philosophers have intended this consciously or unconsciously;
and if one could show historically that some philosophers
didn't have this objective,
then one can offer a philosophical proof that,
to the extent they wished to exist (as philosophers),
they must have intended it.
Because merely apprehending multiplicity,
as such in its factical occurrence, is history.
Whoever wants this alone as the absolute one
intends that nothing should exist except history.
If this person now says that there is something in addition to history,
which he wants to designate by the name “philosophy,”
then he contradicts himself and thereby destroys his entire statement.

Since, as a result, absolutely all philosophical systems must agree,
to the extent that they wish to exist on their own apart from history,
the difference between them, taken initially in a superficial and
historical manner, can only reside in what they propose as oneness,
the one true self-contained in-itself
(e.g., the absolute; therefore one could say in passing that
the task of philosophy could be expressed as
the presentation of the absolute).

In this way, I say, various philosophies could be differentiated
if one looked at them superficially and historically.
But let's go further.
I claim that, to the extent that general agreement is possible
among actually living individuals in regard to any manifold,
to that extent the oneness of principle is in truth and in fact one.
For divergent principles become divergent results,
and consequently yield thoroughly divergent and mutually incoherent worlds,
so that no sort of agreement about anything is possible.
But if one principle alone is right and true,
it follows that only one philosophy is true,
namely the one which makes this one principle its own,
and all others are necessarily false.
Therefore, in case there are several philosophies simultaneously presenting
different absolutes, either all, or all except one, are false.

Further, this significant consequence also follows:
since there is only one absolute, a philosophy which has not
made this one true absolute its own simply doesn't have
the absolute at all but only something relative,
a product of an unperceived disjunction which for this very reason
must have an opposing term.
Such a philosophy leads all multiplicity back, not to absolute oneness as
the task requires, but only to a subordinate, relative oneness;
and thus it is refuted and shown in its insufficiency
not just by the true philosophy but even by itself,
if only one is acquainted with philosophy's true task and
reflects it more prudently than this system does.
Therefore the entire differentiation of philosophies according to
their principles of oneness is only provisional and historical,
but cannot in any way be adequate by itself.
However, since we must start here with provisional and historical knowledge,
let's return to this principle of classification.
Again, the science of knowing may be one of the possible philosophies.
Therefore, if it makes the claim, as it already has,
that it resembles no previous system but is
completely distinct from them all, new, and self-sufficient,
then it must have a different principle of oneness from all the rest.
What did these have as a principle of oneness?
In passing, let me note that it is not my intention here
to discuss the history of philosophy and to let myself in
for all the controversies which would be aroused for me in this way,
instead I simply intend to progress gradually in developing my own concept.
For this purpose, what I will say could serve as well
if it were assumed arbitrarily and were not historically grounded,
as if it were historically true.
This can be abundantly demonstrated if such demonstration is necessary
and [if] there are people concerned about this.
I claim that this much is evident from all
philosophies prior to Kant,
the absolute was located in being, that is,
in the dead thing as thing.
The thing should be the in-itself.
(In passing: I can add that, except for the science of knowing,
since Kant, philosophers everywhere without exception,
the supposed Kantians as well as the supposed
commentators on and improvers of the science of knowing,
have stayed with the same absolute being,
and Kant has not been understood in his true,
but never clearly articulated principle.
Because it is not a matter of what one calls being,
but of how one grasps and holds it inwardly.
For all that one names it [the absolute] “I,”
if one fundamentally objectifies it,
and separates it from oneself,
then it is the same old thing-in-itself).
But surely everyone who is willing to reflect can perceive that
absolutely all being posits a thinking or consciousness of itself;
and that therefore mere being is always only one half of a whole
together with the thought of it, and is therefore
one term of an original and more general disjunction,
a fact which is lost only on the unreflective and superficial.
Thus, absolute oneness can no more reside in being than in its
correlative consciousness;
it can as little be posited in the thing as
in the representation of the thing.
Rather, it resides in the principle,
which we have just discovered, of
the absolute oneness and indivisibility of both,
which is equally, as we have seen,
the principle of their disjunction.
We will name this principle pure knowing, knowing in itself, and,
thus, completely objectless knowing, because otherwise
it would not be knowing in itself but
would require objectivity for its being.
It is distinct from consciousness,
which posits a being and is therefore only a half.
This is Kant's discovery, and is what makes him
the founder of Transcendental Philosophy.
Like Kantian philosophy, the science of knowing is
transcendental philosophy,
and thus it resembles Kant's philosophy in that it does not posit
the absolute in the thing, as previously, or
in subjective knowing, which is simply impossible,
because whomever reflects on this second term already has the first,
but in the oneness of both.

But now, how does the science of knowing
differentiate itself from Kantianism?
Before I answer, let me say this.
Whoever has caught a genuine inner glimpse of just
this higher oneness has already achieved in this first hour
an insight into the true home for
the principle of the sole true philosophy,
which is nearly entirely lacking in this philosophical era;
and he has acquired a conception of the science of knowing
and an introduction to understanding it,
which has also been wholly lacking.
Namely, as soon as one has heard that
the science of knowing presents itself as idealism,
one immediately infers that it locates the absolute
in what I have been calling thinking or consciousness
which stands over against being as its other half and
which therefore can no more be the absolute than can its opposite.
Nevertheless, this view of the science of knowing is accepted
equally by friends and enemies and there is no
way to dissuade them from it.

In order to find a place for their superiority at improving things,
the improvers have switched the absolute from the term in which,
according to their view, it resides in the science of knowing
to the other term to which they append in addition the little word “I”
which may well be the single net result of Kant's life
and, if I may name myself after him, of my life as well,
which has also been devoted to science.

\subsection*{Second Lecture (Wednesday, April 18, 1804)}
Honorable Guests:

We will begin today's lecture with
a brief review of the previous one.
In conducting this review I have an ulterior motive,
namely to adduce what in general can be said about
the technique of fixing lectures like this in memory
and of reproducing them for oneself,
and to discuss the extent to which
transcribing is or is not useful.
“In general,” I say, because in what concerns memory
and the possibility of directing the attention
simultaneously to a number of objects,
one finds a great discrepancy among people;
and I in particular am among the least fortunate in this regard,
since I am utterly lacking in what is usually called memory,
and my attention is capable of taking up no more than one thing at a time.
For this reason my recommendations are all the less authoritative,
and each of you must decide for yourselves how far they apply to you
and how you might employ them.

For me, the proper and favorite listener would be the one who is able to
reproduce the lecture for himself at home, not literally
for that would be mechanical memory, but by pondering and reflection;
and indeed following the course of the argument absolutely freely,
moving backwards from the results with which we concluded to the premises,
forwards from the premises with which we started deducing the result,
and moving both ways at once from the center.
Further, this listener should be able to do so in absolute independence
from specific modes of speech; and, since we propose to conduct
what is really only one whole, self-contained presentation of
the science of knowing in many lectures separated by hours and days,
and with the single lecture periods composing the integrative parts
of this whole, just as the single minutes of the lecture hour make up its parts,
once again my favorite listener will be the one who is able to present
all the single lecture sessions comprehensively as a whole,
whether beginning with the first, or with the last he has heard,
or with any of those in the middle.
This is the first point.

Now, in the second place, what is most noteworthy in each lesson
for each person, and therefore what each grasps most surely, is
whatever new element that person learns and clearly understands.
What we genuinely comprehend becomes part of ourselves, and
if it is a genuinely new insight, it produces a personal transformation.
It is impossible that one not be, or that one cease to be,
what one has genuinely become, and for exactly this reason
the science of knowing can, more than any other philosophy,
commit itself to reactivate the dormant instinct for thinking,
because it introduces new concepts and insights.

What was new in the last hour for those who are not familiar with
the science of knowing, and which may have appeared in a new light
for those already acquainted with it, was this:
if one reflects properly, absolutely all being presupposes
a thought or consciousness having that being as its object;
that consequently being is a term of a disjunction,
the other half of which is thinking;
and that for this reason oneness is not to be found
in the one or the other but in the connection of both.
Oneness thus is the same as pure knowing in and for itself,
and therefore it is knowledge of nothing;
or in case you find the following expression easier to remember,
it (oneness) is found in truth and certainty,
which is not certainty about any particular thing,
since in that case the disjunction of being and knowing is already posited.
So if, in the effort to reproduce the first lecture from within,
someone had clearly and vividly hit upon just this single point,
then it would have been possible with a little logical reflection
to develop all the rest from it.
For example, the listener might inquire:
How did we get to the point of proving that
being had a correlative term?
Did this arise in a polemical context?
Was being taken to be absolute oneness
rather than a correlative term?
Then each would have recalled that this was the case up until Kant.
He could then have asked himself:
But how generally did we come to investigate
what might or might not be absolute oneness?
Thus each would recall just from knowing
why he attended this lecture series that
we were supposed to be doing philosophy and
that the essence of philosophy was said to be
the assertion of absolute oneness
and the reduction of all multiplicity to it,
and so the whole thought process
can be laid out without any problem:
what is the science of knowing, etc.

But this restoration must not lack depth and thoroughness.
For example, “reducing multiplicity to oneness” is
a brief formula, easy to remember, and it is comfortable to use it
in answering the common question “What is philosophy?”
which is a question to which one usually doesn't quite know how to respond.
“But,” one asks oneself, “do you really know what you are saying:
can you clarify it to yourself to the extent of
providing a lucid and transparent construction of it?
Has it been described? How has it been described?
With such and such words. All right, the lecturer said this:
and these are words! I will construct it.”
Or, this thing which is neither being nor consciousness but rather
the oneness of both, which has been presented as
the absolute oneness of the transcendental philosophy,
can be indicated with these words.
But this cannot yet be completely clear and transparent to you,
because all philosophy is contained in the transparency of this oneness,
and from here on we will do nothing else but work on
heightening the clarity of this one concept,
in which at one stroke I have given you the whole.
If it were already completely clear to you,
you would not need me any longer.
Nevertheless, I would add that everyone must possess
more than the bare empty formula.
One must also have a living image of this oneness
which is firmly fixed and which never leaves him.
With my lectures I attend to this, your fixed image;
we will extend and clarify it together.
If someone does not have this image,
there is no way for me to address them, and
for this person my whole discourse becomes talk about nothing,
since in fact I will discuss nothing except this image.

And so that I finally say directly what it all comes down to:
without this free, personal recreation of the exposition of
the science of knowing in its living profundity,
which I have just mentioned,
one will have no possible use for these lectures.
“The subject cannot remain simply in the form
in which I express myself here”;
even though you of course can recall it by, and out of,
yourselves in just this form.
In short, a middle term must come between my act of exposition
and your active mastery of what has been expounded,
and that term is your own rediscovery;
otherwise, everything ends with the act of exposition and
you never attain active mastery.

It is quite unimportant whether one undertakes
this recreation with pen in hand
(as I, for example, would do it because I have no memory
but rather an imagination,
which can only be held in check by written letters)
or whether someone with more memory and a tamer imagination
does it in free thought.
It is essential only that each one does it
in the individually appropriate way;
and, in no case can recreating it in writing cause harm.

In the light of what has been said,
whatever is noted down during the lecture
cannot take the place of a proper recreation;
instead it can only serve as resource for the act,
which must be undertaken with or without this assistance.
With the deliberate speech, considerable pauses after important
paragraphs, and the repetitions of significant phrases
which I am using, it must certainly be possible for
your pens to catch in passing the main points of the
lecture for the required task.
For my part, if I had to attend lectures like these,
I would not even begin to try to note down more than such main points,
since while writing I cannot listen energetically,
and while listening energetically I cannot write;
for me it would be more a matter of the whole living lecture rather than
the isolated, dead words, and more especially a matter of
the seldom noticed, but very genuine and real,
physical/spiritual action of keen thinking carried out in my presence.
Yet I am completely certain that it can be quite different with others
and that more easily activated spirits could indeed do
two things at the same time equally well.

So much once and for all on this topic!
Now let's carry on with the investigation begun last time,
that is, with provisionally answering the question:
What is the science of knowing?

All transcendental philosophy, such as Kant's
(and in this respect the science of knowing is not
yet different from his philosophy),
posits the absolute neither in being nor in consciousness
but in the union of both.
Truth and certainty in and for itself = A.
Thus it follows (and this is another point
through which today's talk links up with the previous one and
by means of which in the general reproduction of all the lectures,
the previous one is to be produced from it and it from the previous one);
it follows, I say, that for this kind of philosophy
the difference between being and thinking,
as valid in itself, totally disappears.
Indeed, everything that can arise in such a philosophy is
contained in the epiphany which we already consummated in ourselves
during the last lecture;
in the insight that no being is possible without thinking, and vice versa—
wholly being and thinking at the same time, and
nothing can occur in the manifest sphere of being
without simultaneously occurring in the manifest sphere of thinking,
if one just considers rightly rather than simply dreaming, and vice versa.
Thus in the vision, which is given and granted,
there is nothing primarily of concern for us as transcendental philosophers.
But according to our insight that the absolute is
not a half but indivisible oneness,
an insight which reaches beyond all appearance,
it is absolutely and in itself neither being nor thinking, but rather:
A
/ \
A.—B T—

If now, in order to apply what has just been said and
to make it even clearer it is assumed that,
in addition to its absolute, fundamental division into B and T,
A also divides itself into x, y, z, then it follows:
1. that everything together in and for itself,
including absolute A, and x, etc., is only just a modification of A;
from this follows directly
2. that it all must occur in B as well as in T.

Assume also that there is a philosophical system that has no doubts that
the dichotomy of B and T, which arises from A, is mere appearance, and
that therefore it would be a genuine transcendental philosophy;
but assume further that this philosophy remained caught in such
an absolute division of A into x, y, z, just as we proposed.
Thus, this system, for all its transcendentalism,
would still not have penetrated through to pure oneness,
nor would it have completed the task of philosophy.
Having eluded one disjunction, it would have fallen into another;
and despite all the admiration one must show it
for first discovering the primordial illusion,
nevertheless with the discovery of this new disjunction,
it is refuted as the true, fully accomplished, philosophy.

The Kantian system is exactly this very one,
precisely characterized by the outline I have just given.
If Kant is studied, not as the Kantians without exception have studied him
(holding on to the literal text, which is often clear
as the heavens, but is also often clumsy, even at important times),
but rather on the basis of what he actually says,
raising oneself to what he does not say
but which he must assume in order to be able to say what he does,
then no doubt can remain about his Transcendentalism,
understanding the word in the exact sense I have just explained.
Kant conceived A as the indivisible union of being and thinking.

But he did not conceive it in its pure self-sufficiency in and for itself,
as the science of knowing presents it, but rather only
as a common basic determination or accident of its
three primordial modes, x, y, z;
(these expressions are meaningful; it cannot be said more precisely)
as a result of which for him there are actually three absolutes and
the true unitary absolute fades to their common property.

The way his decisive and only truly meaningful works,
the three critiques, come before us, Kant has made three starts.
In The Critique of Pure Reason, his absolute (x) is sensible experience,
and in that text he actually speaks in a very uncomplimentary way
about Ideas and the higher, purely spiritual world.
From earlier writings and a few casual hints in this Critique itself,
one might conclude that in his own view the matter didn't end there.
But I would commit myself to showing that these hints are
only one more inconsistency, since if we correctly follow
the implications of the premisses stated in that text,
then the supersensible world must totally disappear,
leaving behind a mere noumenon which has its
complete realization in the empirical world.
Moreover, he has the correct concept of the noumenon,
and by no means the confused Lockean notion
which his followers have imposed on him.
The high inner morality of the man corrected the philosopher,
and he published the Critique of Practical Reason.
In this text the I comes to light as something in itself
through the inherent categorical concept
as it could not possibly do in the Critique of Pure Reason,
which is solely based on, and drawn from, what is empirically;
and thus we get the second absolute, a moral world = z.
Still, not all the phenomena that are undeniably present
in self-observation have been accounted for;
there still remains the notions of
the beautiful, the sublime, and the purposive,
which are evidently neither theoretical cognitions nor moral concepts.
Further, and more significantly,
with the recent introduction of
the moral world as the one world in itself,
the empirical world is lost,
as revenge for the fact that
the latter had initially excluded the moral world.
And so the Critique of Judgment appears,
and in its Introduction,
the most important part of this very important book,
we find the confession that the sensible and supersensible worlds
must come together in a common but wholly unknown root,
which would be the third absolute = y.
I say a third absolute, separate from
the other two and self-sufficient,
despite the fact that it is supposed to be
the connection of both other terms;
and I do not thereby treat Kant unjustly.

Because if this y is inscrutable, then while it may indeed always
contain the connection, I at least can neither comprehend it as such,
nor collaterally conceive the two terms as originating from it.
If I am to grasp it, I must grasp it immediately as absolute,
and I remain trapped forever, now as before, in the
(for me and my understanding) three absolutes.
Therefore, with this final decisive addition to his system,
Kant did not in any way improve that which we owe to him,
he only generously admitted and disclosed it himself.
Let me now characterize the science of knowing in this historical movement,
from which, to be sure, my speculations, which are independent from Kant,
take their origin.

Its essence consists first in discovering the root (indiscernible for Kant)
in which the sensible and supersensible worlds come together
and then in providing the actual conceptual derivation of both worlds
from a single principle.
The maxims which Kant so often repeated orally and in writing,
which his followers parroted
(we must stop short again and cannot go farther),
and which pre-Kantian dogmatists, too,
could have used to answer Kant
(we must stand by our dogmatism and cannot go farther),
are here completely rejected as maxims of weakness or inertia,
which are then taken to apply to everyone.
The science of knowing's own maxim is
to admit absolutely nothing inconceivable and
to leave nothing unconceived;
and it is satisfied to wish not to exist
if something is pointed out to it which it hasn't grasped,
since it will be everything or nothing at all.
To avoid all misunderstanding let me add that
if it too must finally admit something inconceivable,
then at least it will conceive it as just what it is,
absolutely inconceivable, and as nothing more;
and thus too it will conceive the point at which
absolute conceiving is able to begin.

Let this much suffice as an historical characterization of
the science of knowing vis-à-vis the sole neighbor
against which it is immediately juxtaposed
and to which it can be compared:
Kantian philosophy.
It cannot be directly compared to any
earlier philosophies or recent afterbirths,
since it shares nothing with them and is totally different.
It shares the common genus of transcendentalism with Kantianism alone,
and to that extent must demarcate itself from it,
a demarcation which has to do simply with the clarity of this property
but not at all with its vain reputation.

Let me give this characterization in pure concepts, higher than,
and independently of, historical setting, and present its schema:
A is admitted; it divides itself both into B and T and
simultaneously into x, y, z.
Both divisions are equally absolute;
one is not possible without the other.
Therefore, the insight with which the science of knowing begins and
which constitutes its distinction from Kantianism is not at all
to be found in the insight into the division into B and T
which we already also completed in the last lecture,
nor in the insight into the division of x, y, z
which we have still not finished but have problematically presupposed;
rather it is to be found in the insight into
the immediate inseparability of these two modes of division.
Therefore the two divisions cannot be seen into immediately,
as it seemed from the outset until now;
instead they can be seen into only mediately
through the higher insight into their oneness.

I call the attention, especially of the returning listeners,
to this most important clue;
there you have in its full simplicity
a characterization of our speculation much earlier
and right from the beginning which did not come into
our first lecture series until the middle of our work.
(This schema is a summary of the whole lecture.
How to recreate the whole out of it is indicated above.)

\subsection*{Third Lecture (Thursday, April 19, 1804)}
Honorable Guests:

First let me clarify a point raised at the end of the last lecture,
which might occasion misunderstanding.
Absolute A is divided into B and T and into x, y, z; all at one stroke.
As into x, y, z, so into B and T; as into B and T, so into x, y, z.
But how have I expressed myself just now?
Once commencing with x and the other time commencing with B.
This is just a perspective, a bias in my speaking.
I certainly know, and even expressly assert,
that implicitly, beyond the possibilities of
my mode of expression and my discursive construction,
both are totally identical, completely comprehended
in one self-contained stroke.
Therefore, I am constructing what cannot be constructed,
with full awareness that it cannot be constructed.

Let me continue now to characterize the science of knowing
on the basis of the indicators found in comparing it
with Kantian transcendentalism,
among other things, I said that Kant very well understood A as
the link between B and T, but that he did not grasp it
in its absolute autonomy.
Instead, he made it the basic common property and accident of three absolutes.
In this the science of knowing distinguishes itself from him.
Therefore this science must hold that knowing (or certainty),
as soon as we have characterized it as A,
must actually be a purely self-sustaining substance;
that we can realize it as such for ourselves;
and it is in just this realization that the genuine realization
of the science consists.
(A “genuine realization,” I say, with which we aren't yet concerned here,
since we are still occupied with stating the simple concept,
which is not the thing itself.)
This constitutes today's thesis.

To begin with, we can demonstrate immediately that
knowing can actually appear as something standing on its own.
I ask you to look sequentially at your own inner experience:
if you remember it accurately, you will find the object and
its representation, with all their modifications.
But now I ask further:
Do you not know in all these modifications;
and is not your knowing, as knowing,
the same self-identical knowing
in all variations of the object?
As surely as you say, “Yes,” to this inquiry
(which you will certainly do if you have carried out the given task),
so surely will knowledge manifest and present itself to you (as = A)
whatever the variation of its objects
(and hence in total abstraction from objectivity);
remaining as (= A); and thus as a substantive,
as staying the same as itself through all change in its object;
and thus as oneness, qualitatively changeless in itself.
This was the first point.

Thus it presents itself to you with impressive,
absolutely irrefutable manifestness.
You understand it so certainly that you say:
“It is absolutely thus, I cannot conceive it differently.”
And if you were asked for reasons, you could
refuse the request and still not give up this contention.
It is manifest to you as absolutely certain.
During all possible variation in the object, you have said,
knowing always remains self-identical.
Have you then run through and exhausted all possible changes in object,
testing in each case whether knowing remains the same?
I do not think so, because how could you have done it?
Therefore, this knowing manifests itself independently of such experiments
and completely a priori as self-sustaining and self-identical
independent from all subjectivity and objectivity.

1. Now, note what actually belongs to this substantial knowing, so conceived,
and do so with the deepest sincerity of self-consciousness,
so that the erroneous view of the science of knowing
which was criticized at the end of our first lecture,
namely that it locates the Absolute in the knowing which
stands over against its object, doesn't arise again here.
It is true that in our experiment we have started with this consciousness
or presentation of an object. T.B.-T.B. and so on.
In this part of the experiment, B made the T different in every new moment,
because the T was altogether nothing else than the T for this B,
and disappears with it.
Now when we raised ourselves to the second part by asking,
“Is not knowing one and the same through out?” and
finding it to be so, we raised ourselves above all
differences of T as well as of B.
Therefore we could express ourselves much more accurately and precisely:
knowing, which for this reason is not subjective, is
absolutely unalterable and self-identical not just independently from all
variability of the object, but also independently from all variability of the
subject without which the object doesn't exist.
The changeable is nothing further, neither the object nor the subject,
but just the mere pure changeable, and nothing else.
Now this changeable in its continuing changeability,
which is itself unchangeable, divides itself into subject and object;
and the purely unchangeable, in which the division of subject and
object falls away, as does change, opposes itself to the changeable.

2. Here has been disclosed a splendid example of an insight that comes from
exhaustive, continuous searching, which cannot be derived from experience,
but which rather is absolutely a priori.
And so past experiences obligate me
to entreat everyone here to whom this insight has really been evident
(which I think is the case with all of you given the simplicity of the task)
to keep this very example in mind, to hold on to it, and,
if the old empiricist demon shows up,
to attack it and send it away promptly,
a until we succeed in completely exterminating it.
I would very much like to be spared the eternal struggle
about whether in general there is manifestness or something a priori
(for both are the same).
Individuals come to the conclusion that this is the case
only by producing it somehow or other in themselves.
This has happened here today and I simply ask that you not forget it.

Result: knowing, in the mentioned meaning as A, has actually appeared
to us as self-sufficient, as independent of all changeability,
and as self-same and self-contained oneness, as was presupposed by
previous historical reports of the science of knowing.
We therefore already seem to have realized
the principle of the science of knowing in ourselves
and to have penetrated into it.

The second advance in today's lecture is this.
We only seem to have done this, but this is an empty seeming.
We see merely that it is so, but we do not see into
what it authentically is as this qualitative oneness.
Precisely because we see into only such a that,
we are trapped in a disjunction and thus in two absolutes,
changeability and unchangingness, to which we
might possibly append a third, the undiscoverable root of both,
and thus end up in the same shape as Kant's philosophy.
The ground of this duality, insurmountable in this way, is as follows:
the that must seem self-creating, just as our recent insight seemed to be.
But this appearance is possible only under the
condition that a point of origin appears,
which seems (as opposed to this self-creation) to be produced by us,
just as the first part of our previously conducted experiment actually
and in fact appeared to be.
In a word, we grasp both changeability and unchangingness equally, and
are inwardly torn into two or three immediate terms.
How then is this to be?
Obviously, it is clear without any further steps both that
one of the two would have to be grasped mediately,
and that this term which is grasped mediately cannot be
unchangingness, which as the absolute can only be realized absolutely,
but rather must be changeability.
The unchangeable would have to be intuited not only in its being,
which we have already done, but instead it would have
to be penetrated in its essence, its one absolute quality.
It (the unchangeable) would have to be worked through in such a way
that changeability would be seen as necessarily proceeding from it
and as mediately graspable only by means of it.

Briefly, clearly, and to fix the point easily in memory:
the insight that knowing is a self-sustaining qualitative oneness
(an insight which is purely provisional and belongs to
a theory of the science of knowing) leads to the question,
“What is it [i.e., knowing] in this qualitative oneness?”
The true nature of the science of knowing resides in answering this question.
In order to analyze this even further, it is clear that
for this purpose one must inwardly construct this essence of knowing.
Or, as in this case is exactly the same thing,
this essence must construct itself.

In this constructive act, it is without any
doubt and is what it is as existing;
and, as existing, it is what it is.
Therefore it is clear that the science of knowing and
the knowing that presents itself in its essential oneness
are entirely one and the same;
that the science of knowing and primordially essential knowing
merge reciprocally into one another and permeate each other;
that in themselves they are not different;
and that the difference which we still make here is
only a verbal difference of the exact kind
mentioned at the start of this lecture.
The primordially essential knowing is constructive,
thus intrinsically genetic;
this would be the original knowing or certainty in itself.
Manifestness in itself is therefore genetic.

And with this we have specified the deepest characteristic difference
demarcating the science of knowing from all other philosophies,
particularly from the most similar, the Kantian.
All philosophy should terminate in knowing in and for itself.
Knowing, or manifestness in and for itself, is actively genetic.
The highest appearance of knowing, which no longer expresses its
inner essence but instead just its external existence, is factical;
and since it is still the appearance of knowing, factical manifestness.
All factical manifestness, even if it is the absolute, remains
something objective, alien, self-constructing but not constructed of knowing,
and therefore something inwardly unexplained, which an exhausted speculation,
skeptical of its own power, calls inexplicable.
Kantian speculation ends at its highest point with factical manifestness:
the insight that at the basis of both the sensible and supersensible worlds,
there must be a principle of connection, thus a thoroughly genetic principle
which creates and determines both worlds absolutely.

This insight, which is completely right in itself,
could occur to Kant only as a result of his reason's absolute,
but unconsciously operative law: that it [that is, his reason]
come to a stop only with absolute oneness,
recognize only this as absolutely substantial,
and derive everything changeable from this one.
This basic law of oneness remains only factical for him,
and therefore its object is unexamined, because he allows it
to work on him only mechanically; but he does not bring
this action itself and its law into his awareness anew.
If he did so, pure light would dawn on him and he would
come to the science of knowing.
Kantian factical manifestness is not even the highest kind,
because he lets its object emerge from two related terms and
does not grasp it as we have grasped the highest factical object,
namely as pure knowing;
instead he grasps it with the qualification that
it is the link between the sensible and supersensible worlds.
That is, he does not grasp it inwardly
and in itself as oneness, but as duality;
his highest principle is a synthesis post factum.
Namely, this means a case when by self-observation one discovers
in one's consciousness two terms of a disjunction and,
compelled by reason, sees that they must intrinsically be one,
disregarding the fact that one cannot say how, given this oneness,
they can likewise become two.

Briefly, this is exactly the same procedure by which in our first lecture
we rose from the discovery of the duality of being and thinking to A
as their required necessary connection, in order initially to construct
for ourselves the transcendentalism common to the science of knowing
and Kantianism, but the matter was not to rest there.
Additionally there should be a synthesis a priori
which is equally an analysis,
since it simultaneously provides
the basis for both oneness and duality.
Kant's highest manifestness, I said, is factical,
and not even the highest factical kind.
The highest factical manifestness has been presented today:
the insight into knowing's absolute self-sufficiency,
without any determination by anything outside itself,
anything changeable.
This is contrary to the Kantian absolute,
which is determined by the transition
between the sensible and supersensible.
Since now this presently factical element in science
is itself to become actively genetic and developing,
then, change in general will be grounded in it
as a genesis pure and simple.

But by no means will any particular change be so grounded.
It seems that absolute facticity could be discovered only
by those who have raised themselves above all facticity,
as I have actually discovered it and consistently
made use of it only after discovering the true
inner principle of the science of knowing,
and as I am using it now to lead the audience
from that point forward in the genetic process.
Kant's manifestness is factical,
we ourselves are presently also standing in facticity;
and I add that everywhere in the scientific world
there is no other kind of manifestness except the factical
(namely in the first principles),
except in the science of knowing.

As far as philosophy is concerned,
after conducting the demonstration with Kant,
we can safely omit tests on other systems.
After philosophy, mathematics makes claims to manifestness,
indeed in some of its representatives it takes on airs
by elevating itself above philosophy,
an error which can be excused in the light of
today's philosophical eclecticism.
Now abstracting here completely from the fact
that things are not so wonderful for mathematics
—not even in regard to how it can and should be,
this science must confess that its principles admit
nothing more than factical manifestness,
regardless of the fact that they will become
actively genetic as we proceed.

For let the arithmetician qua arithmetician simply tell me
how he is able to elicit a solid and permanent number one;
or let the geometer explain what fixes and holds his space
for him while he draws his continuous lines through it;
and whether these and ever so many other ingredients
which he needs for the possibility of his derivations are
given in any other way than through factical intuition.
Of course this does not in any way constitute an
objection to mathematics;
as mathematics it can and should be nothing else.
It is certainly not our business to obscure the boundaries of the sciences;
but one should simply recognize, and this science, like all the others,
should know that it is neither the first nor self-sufficient
but rather that the principles of its possibility lie in another,
higher science.
Now, since in the actual sciences generally
no other principles are available than those
that are factically manifest, and since by contrast
the science of knowing intends to introduce
entirely genetic manifestness and then to
deduce the factical from it,
it is clear that essentially in its spirit and life
the science of knowing is wholly different from
all previous scientific employment of reason.

It is clear that it is not known to anyone who has not studied it directly
and that nothing can take the place of such study.
It is equally clear that there is no perspective or premise
which has appeared in previous life or science
from which it can be seen as true, attacked, or refuted,
because whatever this perspective or premise is,
and however certain it might be,
still it is nonetheless surely only factically manifest,
and this science accepts nothing of this kind unconditionally,
but does so only under conditions
which it determines in its genetic analysis.
But whoever wants to argue against the science
of knowing using such a perspective as its principle,
wants unconditional agreement which is already
once and for all ruled out in advance.
Therefore he is arguing from a premise that has not been accepted
{ex non concessis} and makes himself ridiculous.
The science of knowing can only be judged internally,
it could be attacked and refuted only internally,
by pointing out an internal contradiction,
an inner inconsistency or insufficiency.
Therefore such activity must be preceded
by study and comprehension and must begin with that.
Until now to be sure the opposite order has been tried;
first judging and refuting and after that, God willing, understanding.
As a result nothing has ensued except that the blows have
completely missed the science of knowing,
which has remained hidden from view like an invisible spirit,
and they have struck instead the chimeras which
these men have created with their own hands.
Following this, they have gone so far astray with these fantasies
and have spread confusion so extensively that today
it can be expected that they would at least
understand that they are confused!

\subsection*{Fourth Lecture (Friday, April 20, 1804)}
Honorable Guests:

It seems to me that we have succeeded,
even in the prolegomenon,
in gaining a very clear and deep insight
into the scientific form of the science
that we wish to pursue here.
Let us continue the observations
with which we have achieved this insight.

Here are the results so far:
that we have certainly grasped knowing
as unchangeable, self-same and self-sustaining,
beyond all change and beyond the subjectivity-objectivity,
which is inseparable from change.
But this insight was not yet the science of knowing itself,
but rather only its premise.
The science of knowing must still actually construct
this inner, qualitatively unchangeable being,
and as soon as it does this,
it will simultaneously create change,
the second term, as well.

The true authentic meaning of this
simultaneous double construction
of the changeable and the unchangeable
will become completely clear
only when we actually and immediately
carry out the construction,
something that belongs within the sphere
of the science of knowing itself,
certainly not in preliminary reports.
Misunderstandings about this are unavoidable at the beginning.
In order to come as close as possible to
complete accuracy right from the start,
I venture on a question that has already been raised.

On introducing the schema:
A
x y z • B-T
I said that the science of knowing stood in the point.
I have been asked whether it doesn't rather go in A.
The most exact answer is that actually and strictly
it doesn't belong in either of these
but rather in the oneness of both.
By itself A is objective and therefore inwardly dead;
it should not remain so, indeed it should become actively genetic.
The point, on the other hand, is merely genetic.
Mere genesis is nothing at all; but this is not just
mere genesis but the determinate genesis that is required by
the absolute qualitative A; [it is] a point of oneness.

Now to be sure this point of oneness can be realized immediately,
oscillating and expending itself in this point;
and we, as scientists of knowing, are this realization inwardly
(I say inwardly and concealed from ourselves).
But this point can neither be expressed nor reconstructed
in its immediacy, since all expression or reconstruction is
conceiving and is intrinsically mediated.
It is expressed and reconstructed just as we have expressed it
at this moment: namely, that one begins from A and,
indicating that it cannot persist alone,
links it to the point;
or one begins with the point and,
indicating that it cannot persist alone,
links it to A, all the while, to be sure,
knowing, saying, and meaning that
neither A nor the point can exist by itself and
that all our talk could not express the implicit truth,
but instead that the implicitly unreconstructible something
which can only be pictured in an empty and objective image is
the organic oneness of both.

Thus, since reconstruction is conceiving, and
since this very conceiving explicitly abandons its own intrinsic validity,
this is precisely a case of conceiving the inconceivable as inconceivable.
Therefore—and I put this here first as a clarification—
this is a question of the organic split into
B-T and x, y, z, as I explained at the beginning of the last lecture;
because when I speak, I must always put one before the other.
But is it actually so?
No, instead it is exactly the same stroke;
and let me add this as well:
this deeper connection must indeed itself be a result and
a lower expression of the higher one just now described.
Finally, in order to say this in its full meaning
and thereby to make your insight into the science of knowing,
and into knowing as such, much clearer:
secondary knowing, or consciousness, with its whole lawful play
by means of fixed change and the manifold
(within it or outside it),
of sensible and supersensible,
and of time and space, comes to be, in principle,
through this recently noted and demonstrated division,
taken merely as a division and nothing else.
Everything we attribute to the subject, as originating from it,
derives from this.
Because it is clear without further ado that from a particular perspective,
namely the science of knowing's synthetic perspective,
the disjunction must be just as absolute as oneness;
otherwise we would be stuck in oneness
and would never get outside it to changeability.
(Let me note in passing that this is
an important characteristic of the science of knowing
and distinguishes it from Spinoza's system,
which also wants absolute oneness
but does not know how to make a bridge from it to the manifold;
and, on the other hand, if it has the manifold,
cannot get from there to oneness.)

As scientists of knowing, we never escape
the principle of division inwardly and empirically
(by means of what we do and promote);
but we certainly escape it intellectually
with regard to what is valid in itself,
in which very regard the principle of division
surrenders and negates itself.

Or, so that I make the point at which
we have arrived even clearer:
since we actually reason as we have just been doing,
where does our reasoning stand,
if we remain exclusively in consciousness?
(“If we remain exclusively in consciousness,”
I say, because we can also lose ourselves
in the intelligible realm,
and there is even, in its place,
an art of consciously losing oneself in it.)
Manifestly in our construction
by means of the principle of division,
it stands in the place not of that which is to be
valid to the extent it is constructed
but of that which is intrinsically valid;
thus it stands wholly autonomously, as has been said,
between the two principles of oneness and separation,
simultaneously annulling both and positing both.

Thus, the standpoint of the science of knowing,
which stands still in consciousness,
is by no means a synthesis post factum;
but instead a synthesis a priori,
taking neither division nor oneness as given,
but creating both at one stroke.
Once again to adduce an even higher perspective
what is the absolute oneness of the science of knowing?
Neither A nor the point,
but instead the inner organic oneness of both.
Besides this given description of
the point of oneness
is there also another?
None at all, we have seen.
Therefore, this description is
the original and absolutely authentic one.
What are its constituents?
The organic oneness of both is
a construction or a concept,
and indeed the single absolute concept,
abstracted from nothing existing,
since even its own separate existence,
and hence the existence of everything conceptual,
is denied.
Further, the construction as such is denied
by the manifestness of what exists autonomously;
thus even the inconceivable, as the inconceivable
and nothing more, is posited by this manifestness,
posited through the negation of the absolute concept,
which must be posited just so that it can be annulled.

And so:

1. the necessary unification and indivisibility
of the concept and the inconceivable is clearly seen into,
and the result may be expressed thus:
if the absolutely inconceivable is to be
manifest as solely self-sustaining,
then the concept must be annulled,
but to be annulled, it must be posited,
because the inconceivable becomes manifest
only with the negation of the concept.
Supplement: hence inconceivable = unchangeable; concept = change.
Therefore along with the foregoing it is evident that
if the unchangeable is to appear, there must be change.

2. Now to be sure inconceivability is only
the negation of the concept, an expression of its annulment.
Therefore it is something which originates
from both the concept and knowing themselves;
it is a quality transferred by means of absolute manifestness.
Noting this, and therefore abstracting from this quality,
nothing remains for oneness except absoluteness
or pure self-sufficiency in itself.

3. The following consideration makes this
particularly important and relevant:
What is pure self-sufficient knowing in itself?
The science of knowing has to answer this question,
or, as we put it more precisely,
it has to construct the presupposed inner quality of knowing.
We are undertaking this construction here:
negating the concept by means of manifestness,
and thus the self-creation of inconceivability is
this living construction of knowing's inner quality.
This inconceivability itself originates in the concept and
in pure immediate manifestness;
likewise the whole quality of the absolute,
as well as the fact that a quality can be applied to it at all,
originates in the concept.
The absolute is not intrinsically inconceivable,
since this makes no sense;
it is inconceivable only when the concept itself tries for it,
and this inconceivability is its only property.
Having recognized this inconceivability as
an alien quality introduced by knowing, I said before,
only pure self-sufficiency, or substantiality, remains in the absolute;
and it is quite true that at best this
self-sufficiency does not originate in the concept,
since it enters only with the latter's annulling.

But it is clear that this quality enters only
within immediate manifestness, within intuition,
and thus is only the representative and correlate of pure light.
This latter is its genetic principle by which, first of all,
according to our hypothesis all manifestness opens up
into genetic manifestness, since pure light
manifests itself implicitly as genesis.
Secondly, the previously presented relationship of
concept to being and vice versa is further determined as follows:
If there is to be an expression and realization of the absolute light,
then the concept must be posited,
so that it can be negated by the immediate light,
since the expression of pure light consists just in this negation.
But the result of this expression is being in itself, period.
[This result] is inconceivable precisely because
pure light is simultaneously destruction of the concept.
Thus, pure light has prevailed as the one focus and
the sole principle of both being and the concept.

4. From the preceding it follows that this inconceivable,
as the bearer of all reality in knowing,
which we grasp in its principle,
is absolute only as inconceivable,
and cannot be thought in any other way.
No other additional hidden qualities are attributable to it.
Just as little can any quality be added to light,
beyond the previously mentioned characteristic,
namely that it annuls the concept and remains absolute being.
If we made such additions, we would,
as Kant has been criticized for doing,
run up against something unexplained and perhaps inexplicable.
As support for this contention,
notice that we have understood it as inconceivable
purely in its form and nothing more.
We have no right to assert anything before
we have seen into it.

So if we posit some other hidden quality,
we have either invented it, or better,
since pure invention from nothing is completely impossible,
we have manufactured it by trying to supply a principle for some facticity.
This happened with Kant when he first factically discovered
the distinction between the sensible and supersensible worlds and
then added to his absolute the additional inexplicable
quality of linking the two worlds,
a move which pushed us back from genetic manifestness
into merely factical manifestness,
completely contravening the inner spirit of the science of knowing.
Therefore, it is important to note that
whatever yet to be determined characteristics
the reality appearing in our knowing may carry in itself,
besides the common basic property of inconceivability,
such characteristics by no means require any new
absolute grounding principle besides the one principle of pure light,
since this would multiply the number of absolutes.
Rather, the multiplicity and change of these various traits is
to be deduced purely from the interaction of the light with itself,
in its multifarious relations to concepts,
and to inconceivable being.

I invite you to the following reflections
so that I can offer a hint about this last point,
raise what has been said today to a higher level,
give the new listeners a unified perspective
for viewing everything that they can learn here,
and give the returning listeners the same perspective
from which they can again gather and reproduce
everything they have heard up to now.
The focus of everything is pure light.
To truly come to this requires that
the concept be posited and annulled
and that an intrinsically inconceivable being be posited.
If it is granted that the light should exist,
then in this judgment everything else mentioned is possible as well.
We have now seen this; it is true; it remains true forever;
and it expresses the basic principle of all knowing.
We can so designate it for ourselves.

Now, however, I want to completely ignore
the content of this insight and reflect on its form,
on our actual situation of insight.
I also think that we, those of us present here
who have actually seen it,
are the ones who had the insight.
As I remember it and as I think we all do,
the process was that we freely constructed the concepts
and premisses with which we began,
that we held them up to each other freely,
and that in holding them together
we were gripped by the conviction that they
belonged together absolutely and formed an indivisible oneness.
Thus we created at least the conditions for the self-manifesting insight,
and so we likewise appeared to ourselves unconditionally.

But let us not go to work with too much haste;
rather let us consider things a little more deeply.
Did we create what we created because we wished to do so,
and therefore as the result of some earlier knowledge,
which we would have created because we wished to create it
as the result of an even earlier knowledge, and so on to infinity,
so that we might never arrive at a first creation?
Somewhere, if the concept is created it must
absolutely and thoroughly create itself,
without anything antecedent
and without any necessity of a “we”;
because this “we,” as has been shown,
always and everywhere requires some previous knowing
and cannot achieve immediate knowing.

Therefore we cannot create the conditions,
they must emerge spontaneously.
Reason must create itself,
independent from any volition or freedom, or self.
But this proposition, disclosed through analysis,
[contradicts] the first which is given reflexively,
and so immediately.
Which one is true, and on which should we rely?
Before trying to answer, let us return again to the matter of the insight
which has become controversial in regard to its principle,
in order to grasp its meaning and true worth clearly.

We realized that if light is to be,
then the concept must be posited and negated.
Therefore, the light itself is not immediately present in this insight,
and the insight does not dissolve into light and coincide with it;
instead it is only an insight in relation to the light,
an insight which objectifies it,
grasping it by its inner quality only.
Thus, whatever the principle and true bearer of this insight might be,
whether we ourselves, as it seems to be,
or pure self-creating reason, as it also seems to be,
the light is not immediately present in this bearer,
instead it is present merely mediately in
a representative and likeness of itself.
First of all, that this light occurs merely mediately applies not
just in the science of knowing but in any possible consciousness
that has to posit a concept so it can annul it;
and the science of knowing rests on
a completely different point
than the one on which
many may have assumed it to rest after the last lecture,
because undoubtedly knowing was understood too simply.

And now to answer the question:
both are clear, therefore both are equally true;
and so, as was said at the start on another occasion,
manifestness rests neither in one nor in the other,
but entirely between both.
We arrive here, and this is the first important and significant result,
at the principle of division,
not as before a division between two terms,
which in that case are to be intrinsically distinct
like A and the point,
but instead a division of something
which remains always inwardly self-same through all division.
In a word, we have to do with constructing and creating
the very same primal concepts which appear one time as immanent,
in the unconditionally evident final being, the I;
and appear the other time as emanent,
in reason, absolute and in itself,
which nevertheless is completely objectified.
Thus, it is a division pure and in itself,
without any result or alteration in the object.

Further, manifestness oscillates
between these two perspectives:
if it is to be really constructed,
then it must be constructed in that way.
Thus, it must be constructed as oscillating
from a to b and again from b to a and
as completely creating both;
thus as oscillating between the twofold oscillation,
which was the first point,
and which gives rise to a three or fivefold synthesis.

What is the common element in all these determinations?
The very same representative of the light,
seen in its familiar inner quality.
Here it stays.
Therefore everything is the same one common consciousness of light.
This consciousness, which is held in common
and therefore cannot be really constructed
but instead can only be thought by means of the science of knowing,
can be regarded or represented in the three or fivefold modifications only
from another standpoint, which this science alone oversees.
